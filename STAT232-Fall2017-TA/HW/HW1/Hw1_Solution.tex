%%%%%%%%%%%%%%%%%%%%%%%%%%%%%%%%%%%%%%%%%
% Structured General Purpose Assignment
% LaTeX Template
%
% This template has been downloaded from:
% http://www.latextemplates.com
%
% Original author:
% Ted Pavlic (http://www.tedpavlic.com)
%
% Note:
% The \lipsum[#] commands throughout this template generate dummy text
% to fill the template out. These commands should all be removed when 
% writing assignment content.
%
%%%%%%%%%%%%%%%%%%%%%%%%%%%%%%%%%%%%%%%%%

%----------------------------------------------------------------------------------------
%	PACKAGES AND OTHER DOCUMENT CONFIGURATIONS
%----------------------------------------------------------------------------------------

\documentclass{article}
\usepackage{kotex}
\usepackage{fancyhdr} % Required for custom headers
\usepackage{lastpage} % Required to determine the last page for the footer
\usepackage{extramarks} % Required for headers and footers
\usepackage{graphicx} % Required to insert images
%\usepackage{subcaption}
\usepackage{amsmath,amsfonts,amsthm,amssymb} % Math packages
\usepackage{centernot}
\usepackage{sectsty} % Allows customizing section commands
\usepackage{newtxtext}
\usepackage[lite,nofontinfo,zswash,straightbraces]{mtpro2}
\usepackage{bm}
% \usepackage{times,mathptmx}
\usepackage{tabto}
\usepackage{titlesec}
\usepackage[shortlabels]{enumitem}
\usepackage{booktabs}
\usepackage{subfig}
\usepackage{cancel}

\DeclareMathOperator*{\argmax}{\arg\!\max}
\newtheorem*{theorem}{Theorem}

\newcommand{\indep}{\rotatebox[origin=c]{90}{$\models$}}
\newcommand\textline[4][t]{%
  \par\smallskip\noindent\parbox[#1]{.333\textwidth}{\raggedright\texttt{+}#2}%
  \parbox[#1]{.333\textwidth}{\centering#3}%
  \parbox[#1]{.333\textwidth}{\raggedleft\texttt{#4}}\par\smallskip%
}

%--------------------------------------------------------------------------------
%KNITR
%---------------------------------------------------------------------------
\usepackage[]{color}
\makeatletter
\def\maxwidth{ %
  \ifdim\Gin@nat@width>\linewidth
    \linewidth
  \else
    \Gin@nat@width
  \fi
}
\makeatother

\definecolor{fgcolor}{rgb}{0.345, 0.345, 0.345}
\newcommand{\hlnum}[1]{\textcolor[rgb]{0.686,0.059,0.569}{#1}}%
\newcommand{\hlstr}[1]{\textcolor[rgb]{0.192,0.494,0.8}{#1}}%
\newcommand{\hlcom}[1]{\textcolor[rgb]{0.678,0.584,0.686}{\textit{#1}}}%
\newcommand{\hlopt}[1]{\textcolor[rgb]{0,0,0}{#1}}%
\newcommand{\hlstd}[1]{\textcolor[rgb]{0.345,0.345,0.345}{#1}}%
\newcommand{\hlkwa}[1]{\textcolor[rgb]{0.161,0.373,0.58}{\textbf{#1}}}%
\newcommand{\hlkwb}[1]{\textcolor[rgb]{0.69,0.353,0.396}{#1}}%
\newcommand{\hlkwc}[1]{\textcolor[rgb]{0.333,0.667,0.333}{#1}}%
\newcommand{\hlkwd}[1]{\textcolor[rgb]{0.737,0.353,0.396}{\textbf{#1}}}%

\usepackage{framed}
\makeatletter
\newenvironment{kframe}{%
 \def\at@end@of@kframe{}%
 \ifinner\ifhmode%
  \def\at@end@of@kframe{\end{minipage}}%
  \begin{minipage}{0.96\columnwidth}%
 \fi\fi%
 \def\FrameCommand##1{\hskip\@totalleftmargin \hskip-\fboxsep
 \colorbox{shadecolor}{##1}\hskip-\fboxsep
     % There is no \\@totalrightmargin, so:
     \hskip-\linewidth \hskip-\@totalleftmargin \hskip\columnwidth}%
 \MakeFramed {\advance\hsize-\width
   \@totalleftmargin\z@ \linewidth\hsize
   \@setminipage}}%
 {\par\unskip\endMakeFramed%
 \at@end@of@kframe}
\makeatother

\definecolor{shadecolor}{rgb}{.97, .97, .97}
\definecolor{messagecolor}{rgb}{0, 0, 0}
\definecolor{warningcolor}{rgb}{1, 0, 1}
\definecolor{errorcolor}{rgb}{1, 0, 0}
\newenvironment{knitrout}{}{} % an empty environment to be redefined in TeX

\usepackage{alltt}
\IfFileExists{upquote.sty}{\usepackage{upquote}}{}
%--------------------------------------------------------------------------------------------------

%Roman Numeral
\makeatletter
\newcommand*{\rom}[1]{\expandafter\@slowromancap\romannumeral #1@}
\makeatother

% Margins
\topmargin=-0.45in
\evensidemargin=0in
\oddsidemargin=0in
\textwidth=6.5in
\textheight=9.0in
\headsep=0.25in 

\linespread{1.1} % Line spacing

% Set up the header and footer
\pagestyle{fancy}
\lhead{\hmwkAuthorName} % Top left header
\chead{\hmwkClass\ (\hmwkClassInstructor\ \hmwkClassTime): \hmwkTitle} % Top center header
\rhead{\firstxmark} % Top right header
\lfoot{\lastxmark} % Bottom left footer
\cfoot{} % Bottom center footer
\rfoot{Page\ \thepage\ of\ \pageref{LastPage}} % Bottom right footer
\renewcommand\headrulewidth{0.4pt} % Size of the header rule
\renewcommand\footrulewidth{0.4pt} % Size of the footer rule

\setlength\parindent{0pt} % Removes all indentation from paragraphs

%----------------------------------------------------------------------------------------
%	DOCUMENT STRUCTURE COMMANDS
%	Skip this unless you know what you're doing
%----------------------------------------------------------------------------------------

% Header and footer for when a page split occurs within a problem environment
\newcommand{\enterProblemHeader}[1]{
\nobreak\extramarks{#1}{#1 continued on next page\ldots}\nobreak
\nobreak\extramarks{#1 (continued)}{#1 continued on next page\ldots}\nobreak
}

% Header and footer for when a page split occurs between problem environments
\newcommand{\exitProblemHeader}[1]{
\nobreak\extramarks{#1 (continued)}{#1 continued on next page\ldots}\nobreak
\nobreak\extramarks{#1}{}\nobreak
}

\setcounter{secnumdepth}{0} % Removes default section numbers
\newcounter{homeworkProblemCounter} % Creates a counter to keep track of the number of problems

\newcommand{\homeworkProblemName}{}
\newenvironment{homeworkProblem}[1][Problem \arabic{homeworkProblemCounter}]{ % Makes a new environment called homeworkProblem which takes 1 argument (custom name) but the default is "Problem #"
\stepcounter{homeworkProblemCounter} % Increase counter for number of problems
\renewcommand{\homeworkProblemName}{#1} % Assign \homeworkProblemName the name of the problem
\section{\homeworkProblemName} % Make a section in the document with the custom problem count
\enterProblemHeader{\homeworkProblemName} % Header and footer within the environment
}{
\exitProblemHeader{\homeworkProblemName} % Header and footer after the environment
}

\newcommand{\problemAnswer}[1]{ % Defines the problem answer command with the content as the only argument
\noindent\framebox[\columnwidth][c]{\begin{minipage}{0.98\columnwidth}#1\end{minipage}} % Makes the box around the problem answer and puts the content inside
}

\newcommand{\homeworkSectionName}{}
\newenvironment{homeworkSection}[1]{ % New environment for sections within homework problems, takes 1 argument - the name of the section
\renewcommand{\homeworkSectionName}{#1} % Assign \homeworkSectionName to the name of the section from the environment argument
\subsection{\homeworkSectionName} % Make a subsection with the custom name of the subsection
\enterProblemHeader{\homeworkProblemName\ [\homeworkSectionName]} % Header and footer within the environment
}{
\enterProblemHeader{\homeworkProblemName} % Header and footer after the environment
}
   
%----------------------------------------------------------------------------------------
%	NAME AND CLASS SECTION
%----------------------------------------------------------------------------------------

\newcommand{\hmwkTitle}{Assignment \#1} % Assignment title
\newcommand{\hmwkDueDate}{Friday,\ Mar\ 17,\ 2017} % Due date
\newcommand{\hmwkClass}{STAT232} % Course/class
\newcommand{\hmwkClassTime}{3:30pm} % Class/lecture time
\newcommand{\hmwkClassInstructor}{Taeryon Choi} % Teacher/lecturer
\newcommand{\hmwkAuthorName}{B. Park \& D. Lim} % Your name

%----------------------------------------------------------------------------------------
%	TITLE PAGE
%----------------------------------------------------------------------------------------
%title_page_1.tex와 연동
%\title{
%\vspace{2in}
%\textmd{\textbf{\hmwkClass:\ \hmwkTitle}}\\
%\normalsize\vspace{0.1in}\small{Due\ on\ \hmwkDueDate}\\
%\vspace{0.1in}\large{\textit{\hmwkClassInstructor\ \hmwkClassTime}}
%\vspace{3in}
%}

%\author{\textbf{\hmwkAuthorName}}
%\date{} % Insert date here if you want it to appear below your name

%----------------------------------------------------------------------------------------

\begin{document}
%\input{./title_page_1.tex}%TITLE PAGE
%\maketitle
\newpage

\begin{minipage}{\columnwidth}
\centering
\textbf{{\LARGE Solutions for Assignment \#1 : Mathematical Statistics}}

\vspace{15pt}
{\Large Beomjo Park \& Daeyoung Lim}

Dept. of Statistics, Korea University
\end{minipage}
%----------------------------------------------------------------------------------------
%	PROBLEM 1
%----------------------------------------------------------------------------------------

% To have just one problem per page, simply put a \clearpage after each problem
\begin{homeworkProblem}
Let $X$ have probability density function (p.d.f.)
\[
  f_X(x) = \left\{ 
  \begin{array}{ll}
  1/4 & 0 < x <1 , \\
  3/8 & 3 < x <5 , \\
  0 & \mbox{otherwise}.
  \end{array}
  \right.
\]

(a) Find the cumulative distribution function of $X$.\\

\begin{problemAnswer}{
  \begin{enumerate}[(a)]
    \item Recall definition and the conditions that a cumulative distribution function $F$ should satisfy. The definition is
    \begin{equation}
      F_{X}(x) = P_{X}(X \leq x) \text{ for all } x
    \end{equation}
    and a function $F$ is a CDF if and only if
    \begin{itemize}
      \item $\lim_{x\to-\infty}F(x)=0$ and $\lim_{x\to\infty}F(x) =1 $
      \item $F(x)$ is a nondecreasing function of $x$
      \item $F(x)$ is right-continuous; that is, for every number $x_{0}$, $\lim_{x\downarrow x_{0}}F(x) = F(x_{0})$.
    \end{itemize}
    Thus, for $x < 1$
    \begin{equation}
    \begin{split}
      F_{X}(x) &= P_{X}(X \leq x) = \int_{0}^{x}1/4\,\mathrm{d}t\\
               &= x/4
    \end{split}
    \end{equation}
    and for $x \in (3,5)$, since the density is not defined at $x=1$ and $x=3$, we should use the right continuity, $\lim_{x\uparrow 1}F_{X}(x)=\lim_{x\downarrow 3}F_{X}(x)$. Therefore, 
    \begin{equation}
     \dfrac{1}{4} = \lim_{x\downarrow 3} \int_{3}^{x}3/8\,\mathrm{d}t + C
     \end{equation} 
    which returns $C = 1/4$. Therefore,
    \begin{equation}
      F_{X}(x) = \begin{cases}
        0, & x \leq 0,\\
        x/4, &0<x<1,\\
        1/4, & 1\leq x \leq 3\\
        (3x - 7)/8, & 3 < x< 5,\\
        1, &x > 5.
      \end{cases}
    \end{equation}
  \end{enumerate}
}
\end{problemAnswer}

\clearpage
(b) Let $Y=1/X$. Find the probability density function $f_Y(y)$ for $Y$.\\

\begin{problemAnswer}{
\begin{enumerate}[(a), start=2]
  \item \emph{(Method 1)} Starting from the cumulative distribution function,
  \begin{equation*}
    \begin{split}
      P(Y\leq y) &= P\left(\dfrac{1}{X}\leq y\right)\\
      &= P\left(\dfrac{1}{y}\leq X\right)\\
      &= \int_{x\geq 1/y} f_{X}(x)\,\mathrm{d}x\\
      &= \begin{cases}
        \displaystyle\int_{1/y}^{1}1/4\,\mathrm{d}x + \int_{3}^{5} 3/8\,\mathrm{d}x, &\text{if $0 < 1/y < 1$}\\
        \displaystyle\int_{1/y}^{5}3/8\,\mathrm{d}x, &\text{if $3 < 1/y < 5$}
      \end{cases}\\
      &= \begin{cases}
        1/2 - 1/(4y),&\text{if $y > 1$}\\
        15/8 - 3 / (8y), &\text{if $1/5 < y < 1/3$}
      \end{cases}
    \end{split}
  \end{equation*}
  Therefore, the probability density function can be computed by differentiating the CDF with respect to $y$, i.e.,
  \begin{equation}
    f_{Y}(y) = \begin{cases}
      1 / (4y^{2}), & y > 1\\
      3 / (8y^{2}), & 1/5 < y < 1/3,\\
      0, & \text{otherwise}
    \end{cases}
  \end{equation}
  


  \emph{(Method 2)}
  Using the change of variable technique given a relation $y = g(x)$,
  \begin{equation}
    f_{Y}(y) = f_{X}(g^{-1}(y))\left|\dfrac{dg^{-1}(y)}{dy}\right|
  \end{equation}
  immediately returns
  \begin{equation}
    f_{Y}(y) = \begin{cases}
      1 / (4y^{2}), & y > 1\\
      3/(8y^{2}),& 1/5<y<1/3,\\
      0, & \text{otherwise}
    \end{cases}
  \end{equation}
  since $g^{-1}(y) = 1/y$ and the Jacobian becomes $1/y^{2}$.
\end{enumerate}
}
\end{problemAnswer}

\end{homeworkProblem}

%----------------------------------------------------------------------------------------
% PROBLEM 2
%----------------------------------------------------------------------------------------

\begin{homeworkProblem}
Find $E[Y(Y-1)]$ for a geometric random variable $Y$ by finding $\frac{d^2}{dq^2} \left( \sum_{y=1}^\infty q^y \right)$. Use this result to find the variance of $Y$. Also check if your answers are correct using the moment generating function.\\

\begin{problemAnswer}{
Note that 
\[
  \frac{d^2}{dq^2} \left( \sum_{y=1}^\infty q^y \right) = \sum_{y=1}^\infty  \left( \frac{d^2}{dq^2} q^y \right) = \sum_{y=1}^\infty y(y-1) q^{y-2}
\]
Therefore
\[
\operatorname{E}[Y(Y-1)] = \sum_{y=1}^\infty y(y-1) q^{y-1}p = pq\times \frac{d^2}{dq^2} \left( \sum_{y=1}^\infty q^y \right) = pq \times \frac{2}{(1-q)^3} = \frac{2q}{p^2}
\]

Now that 
\[
\operatorname{Var} (Y) = \operatorname{E}(Y^2) - \operatorname{E}(Y)^2 = \operatorname{E}(Y(Y-1)) + \operatorname{E}(Y) - \operatorname{E}(Y)^2 = \frac{2q}{p^2} + \frac{1}{p} - \frac{1}{p^2} = \frac{q}{p^2}
\]

Since the moment generating function of $Y$ is
\[
M_Y(t) = \frac{pe^t}{1-(1-p)e^t}, \quad t<-\log(1-p)
\]

Variance of $Y$ can be obtained by
\[
M''_Y(0) - [M'_Y(0)]^2 = \left. \left[ \frac{pe^t}{((p-1)e^t + 1)^2)} - \frac{2(p-1)pe^{2t}}{((p-1)e^t + 1)^3} \right] - \left[ \frac{pe^t}{(p-1)e^t +1)^2} \right]^2 \right|_{t=0} = \frac{q}{p^2}
\]

} \end{problemAnswer}

\end{homeworkProblem}
%----------------------------------------------------------------------------------------
%	PROBLEM 3
%----------------------------------------------------------------------------------------

\begin{homeworkProblem}
Let the density function of a random variable $Y$ be given by
\begin{align*}
  f(y) = \begin{cases}
    \frac{2}{\pi(1+y^2)}, & -1\le y \le 1,\\
    0,  & \text{elsewhere.}
  \end{cases}
\end{align*}

(a) Find the distribution function.\\

\begin{problemAnswer}{
  \begin{enumerate}[(a)]
    \item Using the definition
    \begin{equation}
      \begin{split}
        F_{Y}(y) &= P_{Y}(Y \leq y)\\
        &= \int_{-1}^{y}\dfrac{2}{\pi(1+x^{2})}\,\mathrm{d}x\\
        &= \left.\dfrac{2}{\pi}\arctan x\right|^{y}_{-1}\\
        &= \dfrac{2}{\pi}\left(\arctan y +\dfrac{\pi}{4}\right)
      \end{split}
    \end{equation}
  \end{enumerate}
} \end{problemAnswer}


(b) Find $E(Y)$.\\

\begin{problemAnswer}{
  \begin{enumerate}[(a), start=2]
    \item Using the definition,
    \begin{equation}
      \begin{split}
        \operatorname{E}(Y) &= \int_{-1}^{1}\dfrac{2y}{\pi(1+y^{2})}\,\mathrm{d}y\\
        &= \left.\dfrac{\log(y^{2}+1)}{\pi}\right|_{-1}^{1}\\
        &= 0
      \end{split}
    \end{equation}
  \end{enumerate}
} \end{problemAnswer}

\end{homeworkProblem}

%----------------------------------------------------------------------------------------
% PROBLEM 4
%----------------------------------------------------------------------------------------

\begin{homeworkProblem}
Let $Y$ denote a random variable with probability density function given by
\begin{align*}
  f(y) = (1/2) e^{-|y|}, \quad -\infty < y <\infty.
\end{align*}
Find the moment-generating function of $Y$ and use it to find $E(Y)$.\\

\begin{problemAnswer} {
  Note that the given function is the PDF of the Double Exponential distribution or the Laplace distribution. Using the definition of the MGF,
   \begin{equation}
     \begin{split}
       \operatorname{E}(e^{tY}) &= \int_{-\infty}^{\infty}\dfrac{1}{2}e^{ty-|y|}\,\mathrm{d}y\\
       &= \int_{-\infty}^{0}\dfrac{1}{2}e^{(t+1)y}\,\mathrm{d}y + \int_{0}^{\infty}\dfrac{1}{2}e^{(t-1)y}\,\mathrm{d}y\\
       &= \dfrac{1}{2(1-t)} + \dfrac{1}{2(1+t)}\\
       &= \dfrac{1}{1-t^{2}},\quad |t|<1
     \end{split}
   \end{equation}
   Note that
   \begin{equation}
     \int_{0}^{\infty}\dfrac{1}{2}e^{(t-1)y}\,\mathrm{d}y <\infty
   \end{equation}
   only if $|t|<1$.
   To obtain the expectation, we use the following relation:
   \begin{equation}
     \operatorname{E}(Y) = \left.\dfrac{d}{dt}M_{Y}(t)\right|_{t=0}
   \end{equation}
   which becomes
   \begin{equation}
     \operatorname{E}(Y)=\left.\dfrac{2t}{(1-t^{2})^{2}}\right|_{t=0} = 0
   \end{equation}
}\end{problemAnswer}
\end{homeworkProblem}

%----------------------------------------------------------------------------------------
% PROBLEM 5
%----------------------------------------------------------------------------------------
\begin{homeworkProblem}
Let $(Y_1, Y_2)$ denote the coordinates of a point chosen at random inside a unit circle whose center is at the origin. That is, $Y_1$ and $Y_2$ have a joint density function given by
\begin{align*}
  f(y_1, y_2) = \begin{cases}
    \frac{1}{\pi}, & y_1^2 + y_2^2 \le 1,\\
    0, & \text{elsewhere.}
  \end{cases}
\end{align*}

Find $P(Y_1 \le Y_2)$.\\

\begin{problemAnswer} {
  The problem is equivalent to calculating $P(Y_{1}-Y_{2}\leq 0)$ and since the integration region is a circle on the $xy$-plane, it is more comfortable to use the polar coordinate. The region satisfying
  \begin{equation}
    y_{1}\leq y_{2} \quad \text{and}\quad y_{1}^{2}+y_{2}^{2}\leq 1
  \end{equation}
  is the circle below the line $y_{2}=y_{1}$. Thus,
  \begin{equation}
  \begin{split}
    P(Y_{1}\leq Y_{2}) &= \int_{-3\pi/4}^{\pi/4}\int_{0}^{1}\dfrac{r}{\pi}\,\mathrm{d}r\,\mathrm{d}\theta\\
    &= \int_{-3\pi/4}^{\pi/4}\dfrac{1}{2\pi}\,\mathrm{d}\theta\\
    &= \dfrac{1}{2}
  \end{split}
  \end{equation}
}\end{problemAnswer}

\end{homeworkProblem}

%----------------------------------------------------------------------------------------
% PROBLEM 6
%----------------------------------------------------------------------------------------

\begin{homeworkProblem}
Let $(X,Y)$ have a uniform distribution over the unit square, i.e. the joint p.d.f. of $(X,Y)$ is given by
\begin{align}
	f(x,y) = \begin{cases}
    1 & 0\le x\le 1,\ 0\le y\le 1\\
    0, & \text{otherwise}
  \end{cases}
\end{align}

Find the moment generating function $Z=-\log(X)-\log(Y)$.

\begin{problemAnswer} {
  Note that the given function is the joint PDF of two independent standard uniform random variables, that is, $X,Y \overset{\text{iid}}{\sim}\operatorname{Unif}(0,1)$. Thus, we can use the relations
  \begin{itemize}
    \item $-\log U \sim \operatorname{Exp}(1)$ where $U\sim \operatorname{Unif}(0,1)$.
    \begin{proof}
      Let $g(u) = -\log u$. Then, $g^{-1}(x) = e^{-x}$. By the variable transformation,
      \begin{equation}
        f_{X}(x) = f_{U}(g^{-1}(x))\left|\dfrac{d g^{-1}(x)}{dx}\right| = e^{-x}
      \end{equation}
      which is the density of the exponential distribution with mean $1$.
    \end{proof}
    \item If $E_{1},E_{2}\overset{\text{iid}}{\sim}\operatorname{Exp}(1)$, $E_{1}+E_{2}\sim \operatorname{Gamma}(2,1)$.
  \end{itemize}
  Thus, the MGF of $Z \sim \operatorname{Gamma}(2,1)$ becomes
  \begin{equation}
    \begin{split}
      \operatorname{E}(e^{tZ}) &= \int_{0}^{\infty}ze^{(t-1)z}\,\mathrm{d}z\\
      &= \dfrac{1}{(t-1)^{2}},\quad t<1
    \end{split}
  \end{equation}
}\end{problemAnswer}

\end{homeworkProblem}
%------------------------------------
%\bibliographystyle{apalike}
%\bibliography{Hw1}
\end{document}