\documentclass[11pt]{article}
\usepackage{amsmath,amsfonts,amssymb,amsthm}
\usepackage{color}
% \usepackage{hyperref}
\usepackage[T1]{fontenc}
\usepackage{mathtools}
\usepackage{graphicx}
\usepackage{subcaption}
% \usepackage{kotex}
\usepackage[shortlabels]{enumitem}
\usepackage{tcolorbox}
\usepackage{newtxtext}
\usepackage{natbib}
\usepackage{mathrsfs}
\usepackage[lite,nofontinfo,zswash,straightbraces]{mtpro2}
% \usepackage{lmodern}
\usepackage{setspace} %줄간격 package
\usepackage{bm}
\usepackage{bbm}
\usepackage[normalem]{ulem}
\usepackage{cancel}
\usepackage{algorithm}
\usepackage[noend]{algpseudocode}
\makeatletter
\def\BState{\State\hskip-\ALG@thistlm}
% \xpatchcmd{\algorithmic}{\itemsep\z@}{\itemsep=1ex}{}{}
\makeatother
\let\oldReturn\Return
\renewcommand{\Return}{\State\oldReturn}

%-------------------------------------------------------------------------------------
% Page layout
\addtolength{\textwidth}{1.2in}
\addtolength{\oddsidemargin}{-0.6in}
\addtolength{\textheight}{1.5in}
\addtolength{\topmargin}{-1.0in}
\renewcommand \baselinestretch{1.2}
\parskip = 6pt
% New command
%\newcommand {\AND}{\quad\mbox{and}\quad}
\newcommand {\for}{\quad\mbox{ for }}
\newcommand {\half}{{\tfrac 12}}
% \newcommand {\bm} [1] {\mbox{\boldmath{$#1$}}}
\newcommand{\bs}{\boldsymbol}
\DeclareMathOperator{\Tr}{Tr}
\newtheorem{definition}{Definition}
% \setmainfont{TEX Gyre Termes}
% \setmathfont[bold-style=TeX]{TG Termes Math}%%%%%%%%%%%%%%%%%%%%%%%%%%%%%%%%%%%%%%%%%
% Short Sectioned Assignment
% LaTeX Template
% Version 1.0 (5/5/12)
%
% This template has been downloaded from:
% http://www.LaTeXTemplates.com
%
% Original author:
% Frits Wenneker (http://www.howtotex.com)
%
% License:
% CC BY-NC-SA 3.0 (http://creativecommons.org/licenses/by-nc-sa/3.0/)
%
%%%%%%%%%%%%%%%%%%%%%%%%%%%%%%%%%%%%%%%%%

%----------------------------------------------------------------------------------------
% \usepackage{fourier} % Use the Adobe Utopia font for the document - comment this line to return to the LaTeX default

\usepackage[english]{babel} % English language/hyphenation
\usepackage{booktabs}
\usepackage{newtxtext}
\usepackage{multirow}
% \usepackage{newtxtext}
% \usepackage[hangul]{kotex}

\DeclareMathOperator*{\argmin}{arg\,min}

% \usepackage{lipsum} % Used for inserting dummy 'Lorem ipsum' text into the template

% \usepackage{sectsty} % Allows customizing section commands
% \allsectionsfont{\centering \normalfont\scshape} % Make all sections centered, the default font and small caps

% \usepackage{fancyhdr} % Custom headers and footers
% \pagestyle{fancyplain} % Makes all pages in the document conform to the custom headers and footers
% \fancyhead{} % No page header - if you want one, create it in the same way as the footers below
% \fancyfoot[L]{} % Empty left footer
% \fancyfoot[C]{} % Empty center footer
% \fancyfoot[R]{\thepage} % Page numbering for right footer
% \renewcommand{\headrulewidth}{0pt} % Remove header underlines
% \renewcommand{\footrulewidth}{0pt} % Remove footer underlines
\setlength{\headheight}{13.6pt} % Customize the height of the header

\numberwithin{equation}{section} % Number equations within sections (i.e. 1.1, 1.2, 2.1, 2.2 instead of 1, 2, 3, 4)
\numberwithin{figure}{section} % Number figures within sections (i.e. 1.1, 1.2, 2.1, 2.2 instead of 1, 2, 3, 4)
\numberwithin{table}{section} % Number tables within sections (i.e. 1.1, 1.2, 2.1, 2.2 instead of 1, 2, 3, 4)

\setlength\parindent{0pt} % Removes all indentation from paragraphs - comment this line for an assignment with lots of text

%----------------------------------------------------------------------------------------
% TITLE SECTION
%----------------------------------------------------------------------------------------

\newcommand{\horrule}[1]{\rule{\linewidth}{#1}} % Create horizontal rule command with 1 argument of height
% \linespread{1.4}
\title{ 
\normalfont \normalsize 
\textsc{korea university, department of statistics} \\ [1pt] % Your university, school and/or department name(s)
\horrule{0.5pt} \\[0.05cm]% Thin top horizontal rule
\Large Dynamic Nelson Siegel Model \\ % The assignment title
\horrule{1pt} \\% Thick bottom horizontal rule
}
\author{Daeyoung Lim} % Your name
\date{}
\doublespacing
\begin{document}
\maketitle
\section{Model}
\begin{itemize}
	\item Initial model
	\begin{equation}
	y_{t}(\tau) = L_{t}+\dfrac{1-e^{-\tau\lambda}}{\tau\lambda}S_{t}+\left(\dfrac{1-e^{-\tau\lambda}}{\tau\lambda}-e^{-\tau\lambda}\right)C_{t}+\eta_{t}(\tau)
	\end{equation}
	\item State space model
	\begin{equation}
		\begin{pmatrix}
			y_{t}(\tau_{1})\\
			y_{t}(\tau_{2})\\
			\vdots\\
			y_{t}(\tau_{N})
		\end{pmatrix}=
		\begin{pmatrix}
			1 & (1-e^{-\tau_{1}\lambda})(\lambda\tau_{1})^{-1} & (1-e^{-\tau_{1}\lambda})(\lambda\tau_{1})^{-1}-e^{\lambda\tau_{1}}\\
			1 & (1-e^{-\tau_{2}\lambda})(\lambda\tau_{2})^{-1} & (1-e^{-\tau_{2}\lambda})(\lambda\tau_{2})^{-1}-e^{\lambda\tau_{2}}\\
			\vdots & \vdots & \vdots \\
			1 & (1-e^{-\tau_{N}\lambda})(\lambda\tau_{N})^{-1} & (1-e^{-\tau_{N}\lambda})(\lambda\tau_{N})^{-1}-e^{\lambda\tau_{N}}
		\end{pmatrix}\begin{pmatrix}
			L_{t}\\
			S_{t}\\
			C_{t}
		\end{pmatrix}+ \begin{pmatrix}
			\eta_{t}(\tau_{1})\\
			\eta_{t}(\tau_{2})\\
			\vdots\\
			\eta_{t}(\tau_{N})
		\end{pmatrix}
	\end{equation}
	The evolution of the factors (state equation):
	\begin{equation}
		\begin{pmatrix}
			L_{t}\\
			S_{t}\\
			C_{t}
		\end{pmatrix}= \begin{pmatrix}
			\mu_{1,t}\\
			\mu_{2,t}\\
			\mu_{3,t}
		\end{pmatrix}+\begin{pmatrix}
			\phi_{11} & \phi_{12} & \phi_{13}\\
			\phi_{21} & \phi_{22} & \phi_{23}\\
			\phi_{31} & \phi_{32} & \phi_{33}
		\end{pmatrix}\begin{pmatrix}
			L_{t-1}\\
			S_{t-1}\\
			C_{t-1}
		\end{pmatrix}+\begin{pmatrix}
			\epsilon_{1,t}\\
			\epsilon_{2,t}\\
			\epsilon_{3,t}
		\end{pmatrix}
	\end{equation}
	\item Compact notation (Dynamic linear model)
	\begin{equation}
		\begin{split}
			y_{t} &= \Gamma \bm{\beta}_{t}+\eta_{t}, \quad \eta_{t} \sim N(\mathbf{0},\Sigma)\\
			\bm{\beta}_{t} &= \mu+\mathbf{G}\bm{\beta}_{t-1}+\epsilon_{t},\quad \epsilon_{t}\sim N(\mathbf{0},\Omega)
		\end{split}
	\end{equation}
\end{itemize}
\section{Forward Filtering Backward Sampling (FFBS) algorithm}
\subsection{Kalman Filter}
\begin{algorithm}[H]
  \caption{Kalman Filter}\label{Kalfilt}
  \begin{algorithmic}[1]
    \Procedure{KalmanFilter}{}
    	\State $f_{0|0}\gets  \left(I_{3}-\mathbf{G}\right)^{-1}\mu\quad \left(\because\operatorname{E}(\bm{\beta})=\mu+\mathbf{G}\operatorname{E}(\bm{\beta})\right)$
    	\State $\operatorname{vec}\left(P_{0|0}\right)\gets \left(I_{9}-\mathbf{G}\otimes \mathbf{G}\right)^{-1}\operatorname{vec}(\Omega)\quad  \left(\because\operatorname{Var}(\bm{\beta}) =  \mathbf{G}\operatorname{Var}(\bm{\beta})\mathbf{G}^{T}+\Omega\right)$
    	\For{$t=1:T$}
    	\State $f_{t|t-1}\gets \mu + \mathbf{G}f_{t-1|t-1}$
    	\State $P_{t|t-1}\gets \mathbf{G}P_{t-1|t-1}\mathbf{G}^{T}+\Omega$
    	\State $K_{t}\gets P_{t|t-1}\Gamma^{T}\left(\Gamma P_{t|t-1}\Gamma^{T}+\Sigma\right)^{-1}$
    	\State $\bm{\beta}_{t|t}\gets f_{t|t-1}+K_{t}\left(y_{t}-\Gamma f_{t|t-1}\right)$
    	\State $P_{t|t}\gets \left(I_{3}-K_{t}\Gamma\right)P_{t|t-1} $
    	\EndFor
    \EndProcedure
  \end{algorithmic}
\end{algorithm}
Note: $\operatorname{vec}(ABC)=\left(C^{T}\otimes A\right)\operatorname{vec}(B)$
\subsection{Backward Sampling}
\begin{algorithm}[H]
  \caption{Backward Sampling}\label{backsamp}
  \begin{algorithmic}[1]
    \Procedure{BackwardSampling}{}
    	\State $\bm{\beta}_{T}\sim N\left(f_{T|T},P_{T|T}\right)$
    	\For{$t = (T-1):1$}
    	\State $f_{t+1|t}\gets \mu + \mathbf{G}f_{t|t}$
    	\State $P_{t+1|t}=\mathbf{G}P_{t|t}\mathbf{G}^{T}+\Omega$
    	\State $f_{t|t,\beta_{t+1}}\gets f_{t|t}+P_{t|t}\mathbf{G}^{T}P_{t+1|t}^{-1}\left(\bm{\beta}_{t+1}-f_{t+1|t}\right)$
    	\State $P_{t|t,\beta_{t+1}}\gets P_{t|t}-P_{t|t}\mathbf{G}^{T}P_{t+1|t}^{-1}\mathbf{G}P_{t|t}$
    	\State $\bm{\beta}_{t} \sim N\left(f_{t|t,\beta_{t+1}}, P_{t|t, \beta_{t+1}}\right)$
    	\EndFor
    \EndProcedure
  \end{algorithmic}
\end{algorithm}
\begin{equation}
	\begin{split}
		\begin{pmatrix}
			\bm{\beta}_{t}\\
			\bm{\beta}_{t+1}
		\end{pmatrix}\mid Y^{t} \sim N \left(\begin{pmatrix}
			f_{t|t}\\
			f_{t+1|t}
		\end{pmatrix},
		\begin{pmatrix}
			P_{t|t} & P_{t|t}\mathbf{G}^{T}\\
			\mathbf{G}P_{t|t} & P_{t+1|t}
		\end{pmatrix}\right)
	\end{split}
\end{equation}
% \newpage
% \nocite{*}
% \bibliographystyle{plain}
% \bibliography{AppliedStatisticsBib}
\end{document}