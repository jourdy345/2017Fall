%%%%%%%%%%%%%%%%%%%%%%%%%%%%%%%%%%%%%%%%%
% Structured General Purpose Assignment
% LaTeX Template
%
% This template has been downloaded from:
% http://www.latextemplates.com
%
% Original author:
% Ted Pavlic (http://www.tedpavlic.com)
%
% Note:
% The \lipsum[#] commands throughout this template generate dummy text
% to fill the template out. These commands should all be removed when 
% writing assignment content.
%
%%%%%%%%%%%%%%%%%%%%%%%%%%%%%%%%%%%%%%%%%

%----------------------------------------------------------------------------------------
%	PACKAGES AND OTHER DOCUMENT CONFIGURATIONS
%----------------------------------------------------------------------------------------

\documentclass{article}
\usepackage{kotex}
\usepackage{fancyhdr} % Required for custom headers
\usepackage{lastpage} % Required to determine the last page for the footer
\usepackage{extramarks} % Required for headers and footers
\usepackage{graphicx} % Required to insert images
%\usepackage{subcaption}
\usepackage{amsmath,amsfonts,amsthm,amssymb} % Math packages
\usepackage{centernot}
\usepackage{sectsty} % Allows customizing section commands
\usepackage{newtxtext}
% \usepackage[lite,nofontinfo,zswash,straightbraces]{mtpro2}
\usepackage{bm}
% \usepackage{times,mathptmx}
\usepackage{tabto}
\usepackage{titlesec}
\usepackage[shortlabels]{enumitem}
\usepackage{booktabs}
\usepackage{subfig}
\usepackage{cancel}

\DeclareMathOperator*{\argmax}{\arg\!\max}
\newtheorem*{theorem}{Theorem}
\newtheorem*{proposition}{Proposition}

\newcommand{\indep}{\rotatebox[origin=c]{90}{$\models$}}
\newcommand\textline[4][t]{%
  \par\smallskip\noindent\parbox[#1]{.333\textwidth}{\raggedright\texttt{+}#2}%
  \parbox[#1]{.333\textwidth}{\centering#3}%
  \parbox[#1]{.333\textwidth}{\raggedleft\texttt{#4}}\par\smallskip%
}

%--------------------------------------------------------------------------------
%KNITR
%---------------------------------------------------------------------------
\usepackage[]{color}
\makeatletter
\def\maxwidth{ %
  \ifdim\Gin@nat@width>\linewidth
    \linewidth
  \else
    \Gin@nat@width
  \fi
}
\makeatother

\definecolor{fgcolor}{rgb}{0.345, 0.345, 0.345}
\newcommand{\hlnum}[1]{\textcolor[rgb]{0.686,0.059,0.569}{#1}}%
\newcommand{\hlstr}[1]{\textcolor[rgb]{0.192,0.494,0.8}{#1}}%
\newcommand{\hlcom}[1]{\textcolor[rgb]{0.678,0.584,0.686}{\textit{#1}}}%
\newcommand{\hlopt}[1]{\textcolor[rgb]{0,0,0}{#1}}%
\newcommand{\hlstd}[1]{\textcolor[rgb]{0.345,0.345,0.345}{#1}}%
\newcommand{\hlkwa}[1]{\textcolor[rgb]{0.161,0.373,0.58}{\textbf{#1}}}%
\newcommand{\hlkwb}[1]{\textcolor[rgb]{0.69,0.353,0.396}{#1}}%
\newcommand{\hlkwc}[1]{\textcolor[rgb]{0.333,0.667,0.333}{#1}}%
\newcommand{\hlkwd}[1]{\textcolor[rgb]{0.737,0.353,0.396}{\textbf{#1}}}%

\usepackage{framed}
\makeatletter
\newenvironment{kframe}{%
 \def\at@end@of@kframe{}%
 \ifinner\ifhmode%
  \def\at@end@of@kframe{\end{minipage}}%
  \begin{minipage}{0.96\columnwidth}%
 \fi\fi%
 \def\FrameCommand##1{\hskip\@totalleftmargin \hskip-\fboxsep
 \colorbox{shadecolor}{##1}\hskip-\fboxsep
     % There is no \\@totalrightmargin, so:
     \hskip-\linewidth \hskip-\@totalleftmargin \hskip\columnwidth}%
 \MakeFramed {\advance\hsize-\width
   \@totalleftmargin\z@ \linewidth\hsize
   \@setminipage}}%
 {\par\unskip\endMakeFramed%
 \at@end@of@kframe}
\makeatother

\definecolor{shadecolor}{rgb}{.97, .97, .97}
\definecolor{messagecolor}{rgb}{0, 0, 0}
\definecolor{warningcolor}{rgb}{1, 0, 1}
\definecolor{errorcolor}{rgb}{1, 0, 0}
\newenvironment{knitrout}{}{} % an empty environment to be redefined in TeX

\usepackage{alltt}
\IfFileExists{upquote.sty}{\usepackage{upquote}}{}
%--------------------------------------------------------------------------------------------------

%Roman Numeral
\makeatletter
\newcommand*{\rom}[1]{\expandafter\@slowromancap\romannumeral #1@}
\makeatother

% Margins
\topmargin=-0.45in
\evensidemargin=0in
\oddsidemargin=0in
\textwidth=6.5in
\textheight=9.0in
\headsep=0.25in 

\linespread{1.1} % Line spacing

% Set up the header and footer
\pagestyle{fancy}
\lhead{\hmwkAuthorName} % Top left header
\chead{\hmwkClass\ (\hmwkClassInstructor\ \hmwkClassTime): \hmwkTitle} % Top center header
\rhead{\firstxmark} % Top right header
\lfoot{\lastxmark} % Bottom left footer
\cfoot{} % Bottom center footer
\rfoot{Page\ \thepage\ of\ \pageref{LastPage}} % Bottom right footer
\renewcommand\headrulewidth{0.4pt} % Size of the header rule
\renewcommand\footrulewidth{0.4pt} % Size of the footer rule

\setlength\parindent{0pt} % Removes all indentation from paragraphs

%----------------------------------------------------------------------------------------
%	DOCUMENT STRUCTURE COMMANDS
%	Skip this unless you know what you're doing
%----------------------------------------------------------------------------------------

% Header and footer for when a page split occurs within a problem environment
\newcommand{\enterProblemHeader}[1]{
\nobreak\extramarks{#1}{#1 continued on next page\ldots}\nobreak
\nobreak\extramarks{#1 (continued)}{#1 continued on next page\ldots}\nobreak
}

% Header and footer for when a page split occurs between problem environments
\newcommand{\exitProblemHeader}[1]{
\nobreak\extramarks{#1 (continued)}{#1 continued on next page\ldots}\nobreak
\nobreak\extramarks{#1}{}\nobreak
}

\setcounter{secnumdepth}{0} % Removes default section numbers
\newcounter{homeworkProblemCounter} % Creates a counter to keep track of the number of problems

\newcommand{\homeworkProblemName}{}
\newenvironment{homeworkProblem}[1][Problem \arabic{homeworkProblemCounter}]{ % Makes a new environment called homeworkProblem which takes 1 argument (custom name) but the default is "Problem #"
\stepcounter{homeworkProblemCounter} % Increase counter for number of problems
\renewcommand{\homeworkProblemName}{#1} % Assign \homeworkProblemName the name of the problem
\section{\homeworkProblemName} % Make a section in the document with the custom problem count
\enterProblemHeader{\homeworkProblemName} % Header and footer within the environment
}{
\exitProblemHeader{\homeworkProblemName} % Header and footer after the environment
}

\newcommand{\problemAnswer}[1]{ % Defines the problem answer command with the content as the only argument
\noindent\framebox[\columnwidth][c]{\begin{minipage}{0.98\columnwidth}#1\end{minipage}} % Makes the box around the problem answer and puts the content inside
}

\newcommand{\homeworkSectionName}{}
\newenvironment{homeworkSection}[1]{ % New environment for sections within homework problems, takes 1 argument - the name of the section
\renewcommand{\homeworkSectionName}{#1} % Assign \homeworkSectionName to the name of the section from the environment argument
\subsection{\homeworkSectionName} % Make a subsection with the custom name of the subsection
\enterProblemHeader{\homeworkProblemName\ [\homeworkSectionName]} % Header and footer within the environment
}{
\enterProblemHeader{\homeworkProblemName} % Header and footer after the environment
}
   
%----------------------------------------------------------------------------------------
%	NAME AND CLASS SECTION
%----------------------------------------------------------------------------------------

\newcommand{\hmwkTitle}{Assignment \#3} % Assignment title
\newcommand{\hmwkDueDate}{Wednesday,\ Oct\ 11,\ 2017} % Due date
\newcommand{\hmwkClass}{STAT232} % Course/class
\newcommand{\hmwkClassTime}{3:30pm} % Class/lecture time
\newcommand{\hmwkClassInstructor}{Taeryon Choi} % Teacher/lecturer
\newcommand{\hmwkAuthorName}{B. Park \& D. Lim} % Your name

%----------------------------------------------------------------------------------------
%	TITLE PAGE
%----------------------------------------------------------------------------------------
%title_page_1.tex와 연동
%\title{
%\vspace{2in}
%\textmd{\textbf{\hmwkClass:\ \hmwkTitle}}\\
%\normalsize\vspace{0.1in}\small{Due\ on\ \hmwkDueDate}\\
%\vspace{0.1in}\large{\textit{\hmwkClassInstructor\ \hmwkClassTime}}
%\vspace{3in}
%}

%\author{\textbf{\hmwkAuthorName}}
%\date{} % Insert date here if you want it to appear below your name

%----------------------------------------------------------------------------------------

\begin{document}
%\input{./title_page_1.tex}%TITLE PAGE
%\maketitle
\newpage

\begin{minipage}{\columnwidth}
\centering
\textbf{{\LARGE Solutions for Assignment \#3 : Mathematical Statistics}}

\vspace{15pt}
{\Large Beomjo Park \& Daeyoung Lim}

Dept. of Statistics, Korea University
\end{minipage}
%----------------------------------------------------------------------------------------
%	PROBLEM 1
%----------------------------------------------------------------------------------------

% To have just one problem per page, simply put a \clearpage after each problem
\begin{homeworkProblem}

Let $X_r$ have a $t$-distribution with $r(\geq 3)$ degrees of freedom : $X_r \sim t(r)$. Compute $E(X_r^2)$.\\

\begin{problemAnswer} {
Note that
\[
X_r \equiv \frac{Z}{\sqrt{W / r}}, \quad Z \sim \operatorname{N} (0,1) ~\indep~ W \sim \chi^2(r)
\]
Then,
\begin{align*}
  E(X_r^2) &= E\left( \frac{Z^2}{W / r} \right) = r \times E\left( Z^2 \right) E\left( \frac{1}{W} \right)    &\left( \because Z^2 ~ \indep ~ 1/W \right)\\
       &= r \times  E\left( \frac{1}{W} \right) = \frac{r}{r-2} &\left( \because Z^2 \sim \chi^2 (1) \right)\\
\end{align*}

From the fact that
\begin{align*}
  E\left( \frac{1}{W} \right) &= \int_0^\infty \frac{1}{w} \frac{1}{2^{r/2} \Gamma(r/2)} w^{r/2 - 1} e^{- w/2} dw 
    = \frac{1}{2 \cdot (r/2 - 1)} \underbrace{\int_0^\infty \frac{1}{2^{(r-2)/2} \Gamma((r-2)/2)} w^{(r-2)/2 - 1} e^{- w/2} dw}_{ = 1}\\
    &= \frac{1}{r-2}
\end{align*}
}\end{problemAnswer}

\end{homeworkProblem}

%----------------------------------------------------------------------------------------
% PROBLEM 2
%----------------------------------------------------------------------------------------

\begin{homeworkProblem}
Suppose that $W_1$ and $W_2$ are independent $\chi^2$ distributed random variables with $\nu_1$ and $\nu_2$ degrees of freedom, respectively. According to Definition 7.3,
$$
F = \frac{W_1 / \nu_1}{W_2 / \nu_2}
$$
has an $F$ distribution with $\nu_1$ and $\nu_2$ numerator and denominator degrees of freedom, respectively. Use the preceding structure of $F$, the independence of $W_1$ and $W_2$, and the result summarized in Exercise 7.30(b) to show

(a) $E(F) = \nu_2 / (\nu_2 - 2)$, if $\nu_2 > 2$.\\

\begin{problemAnswer}{

}\end{problemAnswer}

(b) $V(F) = [2\nu_2^2 (\nu_1 + \nu_2 - 2)] / [\nu_1 (\nu_2 - 2)^2 (\nu_2 - 4)]$, if $\nu_2 > 4$.\\

\begin{problemAnswer}{

}\end{problemAnswer}

\end{homeworkProblem}


%----------------------------------------------------------------------------------------
%	PROBLEM 3
%----------------------------------------------------------------------------------------

\begin{homeworkProblem}
Suppose that $Z$ has a standard normal distribution, $Z \sim N(0,1)$, $Y$ has an exponential distribution with mean 2, $Y \sim {\rm Exp}(2)$, and that $Z$ and $Y$ are independent. Let $T=Z/\sqrt{Y/2}$. Find the p.d.f (probability density function) of $T$.\\

\begin{problemAnswer} {

}\end{problemAnswer}

\end{homeworkProblem}


%----------------------------------------------------------------------------------------
% PROBLEM 4
%----------------------------------------------------------------------------------------
\begin{homeworkProblem}
Let $X_1,\ldots,X_n$ be a random sample from a $N(\mu_1,\sigma^2)$ and $Y_1,\ldots,Y_m$ 
be a random sample from a $N(\mu_2,\sigma^2)$, independent with $X_1,\ldots,X_n$.

(a) Define 
\[
 W=\frac{\sum_{i=1}^n (X_i - \overline{X}_n )^2}{\sum_{i=1}^m (Y_i - \overline{Y}_m )^2}
\]
Compute the expected value of $W$, ${\rm E}(W)$. \\

\begin{problemAnswer} {

}\end{problemAnswer}


(b) Let $\displaystyle \overline{X}_{n-1} = \frac{1}{n-1} \sum_{i=1}^{n-1} X_i$, $\displaystyle \overline{Y}_{m} = \frac{1}{m} \sum_{i=1}^{m} Y_i$ and $\displaystyle T_m = \left[\frac{1}{n} \sum_{i=1}^m (Y_i - \overline{Y}_m )^2 \right]^{1/2}$. Determine the value of a constant $k$ such that the random variable $k(X_n-\overline{X}_{n-1} )/T_m$  will have a $t$ distribution with an appropriate degree of freedom.\\

\begin{problemAnswer} {


}\end{problemAnswer}

\end{homeworkProblem}


%----------------------------------------------------------------------------------------
% PROBLEM 5
%----------------------------------------------------------------------------------------
\begin{homeworkProblem}
Let $X_1,\ldots,X_n$ be i.i.d. $N(\mu,\sigma^2)$ with $-\infty < \mu <\infty$ and $\sigma^2>0$ unknown. 
Define $Y= \frac{1}{n} \sum_{i=1}^n (X_i - \overline{X}_n )^2$. Compute the expected value of $Y^2$, ${\rm E}(Y^2)$. \\

\begin{problemAnswer} {

}\end{problemAnswer}

\end{homeworkProblem}


%------------------------------------
%\bibliographystyle{apalike}
%\bibliography{Hw1}
\end{document}